\documentclass{book}

\usepackage{opensourcepress}
\usepackage{opensourcepress-gentoo}

\begin{document}


\section{Zweiter Durchgang}

\begin{osplist}


\item Zwischen einer ueber- und einer untergeordneten Ueberschrift
  sollte immer irgendein Text stehen, damit das nicht so daemlich
  aussieht (Ansprechenderes Layout sorgt auf psychologischer Ebene
  dafuer, dass die Leser/innen mit guter Laune bei der Stange bleiben
  und damit auch selbst besser lernen.)

  \begin{ospcode} 
    {\textbackslash}section\{Vorbereitungen\}

    irgendein Text

    {\textbackslash}subsection\{Noch mehr Vorbereitungen\}

  \end{ospcode}

\item Gentoo-System, Gentoo-Nutzer u.a. Zusammensetzungen mit
  deutschen Wortbestandteilen (duerfen auch Lehnwoerter sein) bitte
  koppeln

\item Es ist immer besser, theoretische Ausfuehrungen um ein
  passendes Beispiel zu ergaenzen. Da muss ich als Leserin naemlich
  nicht mehr versuchen, Deinen abstrakten Text in der Praxis
  fehlertraechtig umzusetzen, sondern sehe sofort: Aha, so muss das
  sein!

\item Bitte die Pseudo-Backus-Naur-Schreibweise der Manpages vermeiden.
Sie verwirrt nur: Ist <> jetzt etwas, was ich mitschreiben soll oder nicht?
Die Manpages sind ja leider auch nicht wirklich konsequent, und wir haben
immerhin mit \emph{} die Moeglichkeit, Platzhalter durch einen anderen Font
kenntlich zu machen.
\end{osplist}

\item Leider gibt es keine Moeglichkeit, \cmd{} und \emph{} (zur
Kenntlichmachung der Platzhalter) zu verschachteln (Hat mit der Schrift im
Endlayout zu tun). Daher muss es
  \begin{ospcode}
  \cmd\{/etc/init.d/net.\}\emph\{interface\}
  \end{ospcode}
hei�en, nicht
  \begin{ospcode}
  \cmd\{/etc/init.d/net.\emph\{interface\}\}
  \end{ospcode}
\end{osplist}

\section{Kapitelabschluss}

Kurze Ablauf�bersicht:

\begin{osplist}
\item Inhaltliche Korrektur
\item Beseitigen von GW, MW, PJ, UW Kommentaren, Beseitigen von FIXME
\item Alle Kommandos durchlaufen
\item Index �berarbeiten, dabei die Textzeile vor Index-Eintraegen
  grundsaetzlich NICHT mit Prozentzeichen abschliessen, da sonst keine
  Leerzeichen im Satz erscheinen
\item Code-Umgebung checken (max. 72 Zeichen, ``{} %'')
\item doppelte Anfuehrungszeichen innerhalb von cmd mit geschweiftem
  Klammerpaar vom Folgenden trennen, also "{}
\item Passiv-Check: grep "w[uei]rd"
\item ispell check
\item make laufen lassen
\item End-Space Check: grep "[^ \\]%$" *.tex | grep -v "\\index" | grep -v "\\label" | grep -v "\}\}%$"
\item Klammer Check: for fl in *.tex;do echo "$fl";echo "----------";sed -n '/begin.ospcode/,/end.ospcode/p' $fl | sed -e 's/\\[a-z]*{\([^{}]*\\)}/\1/' | sed -e 's/\\[a-z]*{\([^}]*\)}/\1/' | sed -e 's/\\[a-z]*{\([^}]*\)}/\1/' | sed -e 's/{\\textasciitilde}//' | sed -e\ 's/{\\textbackslash}//' |  grep ``\([^\\]{\|[^\\]}\)'';done
\item pdf drucken und lesen
\end{osplist}

\end{document}
